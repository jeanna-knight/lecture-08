\documentclass{tufte-handout}

\usepackage{xcolor}

% set hyperlink attributes
\hypersetup{colorlinks}

% set image attributes:
\usepackage{graphicx}
\graphicspath{ {images/} }

% indent subsections
\newenvironment{subs}
  {\adjustwidth{3em}{0pt}}
  {\endadjustwidth}

% ============================================================

% define the title
\title{SOC 4015/5050: Lecture 08 Functions}
\author{Christopher Prener, Ph.D.}
\date{Fall 2018}
% ============================================================
\begin{document}
% ============================================================
\maketitle % generates the title
% ============================================================

\vspace{5mm}
\section{Packages}
\begin{itemize}
\item \texttt{car}
\item \texttt{effsize}
\item \texttt{ggridges}
\item \texttt{ggstatsplot}
\item \texttt{pwr}
\item \texttt{stats}
\item \texttt{tidyverse}
\begin{itemize}
\item \texttt{broom}
\item \texttt{ggplot2}
\item \texttt{readr}
\end{itemize}
\end{itemize}

\vspace{5mm}
\section{Reading and Writing Data}\marginnote{You need to use the \texttt{here} package to build your file paths correctly.}
\begin{subs}
\subsection{Reading Data}
\noindent \texttt{readr::}{\color{red}\texttt{read\_csv}}\texttt{(path = \textit{filePath})}\\

\vspace{3mm}
\subsection{Writing Data}
\noindent \texttt{readr::}{\color{red}\texttt{write\_csv}}\texttt{(\textit{dataFrame}, path = \textit{filePath})}\\
\end{subs}

\vspace{5mm}
\section{Tidy Output}
\noindent \texttt{broom::}{\color{red}\texttt{tidy}}\texttt{(\textit{testFunction})}\\

\vspace{5mm}
\section{Saving Plots}\marginnote{This use of \texttt{ggsave} will save the plot you have created most recently.}
\noindent \texttt{ggplot2::}{\color{red}\texttt{ggsave}}\texttt{(\textit{filename}, dpi = \textit{val})}\\

\vspace{5mm}
\section{Plots for Mean Difference}
\begin{subs}
\subsection{Box Plot}
\noindent \texttt{ggplot2::}{\color{red}\texttt{geom\_boxplot}}\texttt{(mapping = aes(\textit{aesthetic}))}\\

\vspace{3mm}
\subsection{Violin Plot}
\noindent \texttt{ggplot2::}{\color{red}\texttt{geom\_violin}}\texttt{(mapping = aes(\textit{aesthetic}))}\\

\vspace{3mm}
\subsection{Violin Plot with Mean Points}\marginnote{You need to set the base aesthetic mapping in your initial \texttt{ggplot()} call.}
\noindent \texttt{ggplot2::}{\color{red}\texttt{geom\_violin}}\texttt{(mapping = aes(\textit{aesthetic})) +}\\
\noindent \texttt{ggplot2::}{\color{red}\texttt{stat\_summary}}\texttt{(fun.y = mean, geom = "point"))}

\vspace{3mm}
\subsection{Ridge Plot}
\noindent \texttt{ggridges::}{\color{red}\texttt{geom\_density\_ridges}}\texttt{(mapping = aes(\textit{aesthetic}))}\\

\vspace{3mm}
\subsection{Ridge Plot with Transparent Fill}
\noindent \texttt{ggridges::}{\color{red}\texttt{geom\_density\_ridges}}\texttt{(mapping = aes(\textit{aesthetic}),\\alpha = \textit{val})}\\

\vspace{3mm}
\subsection{Stats Plot}\marginnote{You \textit{do not} need to call the \texttt{ggplot()} function first! Valid \texttt{plot.type} values are \texttt{"violin"}, \texttt{"box"}, and \texttt{"boxviolin"}.}
\noindent \texttt{ggstatsplot::}{\color{red}\texttt{ggbetweenstats}}\texttt{(data = \textit{dataFrame}, \\x = \textit{xvar}, y = \textit{yvar}, effsize.type = "biased", \\plot.type = \textit{plotType})}\\
\end{subs}


\vspace{5mm}
\section{Levene's Test}
\noindent \texttt{car::}{\color{red}\texttt{leveneTest}}\texttt{(\textit{yVar \textasciitilde\ xVar}, data = \textit{dataFrame})}\\

\vspace{5mm}
\section{One-Sample T Test}
\noindent \texttt{stats::}{\color{red}\texttt{t.test}}\texttt{(\textit{dataFrame}\$\textit{yVar}, mu = \textit{val})}\\

\vspace{5mm}
\section{Two-Sample (Independent) T Test}\marginnote{Do not forget to adjust the value of \texttt{var.equal} based on the findings of the Levene's test.}
\noindent \texttt{stats::}{\color{red}\texttt{t.test}}\texttt{(\textit{dataFrame}\$\textit{yVar} \textasciitilde\ \textit{dataFrame}\$\textit{xVar}, \\ var.equal = \textit{FALSE})}\\

\vspace{5mm}
\section{Reshaping Data}
\begin{subs}
\subsection{Wide to Long}
\noindent \texttt{tidyr::}{\color{red}\texttt{gather}}\texttt{(\textit{dataFrame}, \textit{key}, \textit{value}, \textit{\ldots})}\\

\vspace{3mm}
\subsection{Long to Wide}
\noindent \texttt{tidyr::}{\color{red}\texttt{spread}}\texttt{(\textit{dataFrame}, \textit{key}, \textit{value})}\\
\end{subs}

\vspace{5mm}
\section{Dependent T Test}
\noindent \texttt{stats::}{\color{red}\texttt{t.test}}\texttt{(\textit{dataFrame}\$\textit{y1}, \textit{dataFrame}\$\textit{y2}, paired = TRUE)}\\

\vspace{5mm}
\section{Cohen's d}
\begin{subs}
\subsection{Independent Observations}
\noindent \texttt{effsize::}{\color{red}\texttt{cohen.d}}\texttt{(\textit{dataFrame}\$\textit{yVar} \textasciitilde\ \textit{dataFrame}\$\textit{xVar}, \\ pooled = \textit{TRUE}, paired = FALSE)}\\

\vspace{3mm}
\subsection{Dependent Observations}
\noindent \texttt{effsize::}{\color{red}\texttt{cohen.d}}\texttt{(\textit{dataFrame}\$\textit{y1}, \textit{dataFrame}\$\textit{y2}, \\ paired = TRUE)}\\
\end{subs}

\vspace{5mm}
\section{Power Analysis}\marginnote{Valid \texttt{type} values are \texttt{"one.sample"}, \texttt{"two.sample"}, and \texttt{"paired"}.}
\noindent \texttt{pwr::}{\color{red}\texttt{pwr.t.test}}\texttt{(d = \textit{val}, power = \textit{val}, sig.level = \textit{val}, \\type = \textit{type}, alternative = "two.sided")}\\


% ============================================================
\end{document}